\documentclass[a4paper,12pt]{article}

%%% Настройка шрифтов
\usepackage{fontspec}

\setmainfont{Times New Roman}
\setsansfont{Arial}
\setmonofont{Courier New}

\defaultfontfeatures{Ligatures=TeX} % Ligatures=TeX чтобы в XeLaTeX работали "--" и "---"

%%% Работа с русским языком

\usepackage{polyglossia}
\setmainlanguage[spelling=modern]{russian}
\setotherlanguage{english}

\usepackage{csquotes} % Ковычки в соответствии с основным языком

%% Включение кириллицы для шрифтов

\newfontfamily{\cyrillicfont}{Times New Roman}
\newfontfamily{\cyrillicfonttt}{Courier New}
\newfontfamily{\cyrillicfontsf}{Arial}

\usepackage{indentfirst}     % Красная строка для всех абзацев

%% Греческие буквы и математические символы согласно русской традиции

\renewcommand{\epsilon}{\ensuremath{\varepsilon}}
\renewcommand{\phi}{\ensuremath{\varphi}}
\renewcommand{\kappa}{\ensuremath{\varkappa}}
\renewcommand{\le}{\ensuremath{\leqslant}}
\renewcommand{\leq}{\ensuremath{\leqslant}}
\renewcommand{\ge}{\ensuremath{\geqslant}}
\renewcommand{\geq}{\ensuremath{\geqslant}}
\renewcommand{\emptyset}{\varnothing}

%%% Дополнительная работа с математикой

\usepackage{amsmath,amsfonts,amssymb,amsthm,mathtools} % AMS
\usepackage{icomma} % "Умная" запятая: $0,2$ --- число, $0, 2$ --- перечисление
\usepackage[labelsep=endash, tableposition=top, figureposition=bottom]{caption}

%%% Работа с картинками

\usepackage{graphicx} % Для вставки рисунков
\graphicspath{{images/}} % папки с картинками
\usepackage{wrapfig}  % Обтекание рисунков текстом

%%% Работа с таблицами

\usepackage{tabularx,tabulary} % Дополнительная работа с таблицами
\usepackage{longtable}         % Длинные таблицы
\usepackage{multirow}          % Слияние строк в таблице
\usepackage{csvsimple}         % Загрузить csv таблицу

\captionsetup[table]{
    justification=centering, % RaggedRight
%    labelsep=endash,
    parskip=0pt,
    singlelinecheck=false,
    skip=0pt,
    aboveskip=0pt,
    belowskip=0pt
}

\captionsetup[figure]{
%    name=Рисунок, % Подпись "Рисунок" а не "рис."
    justification=centering,
%    labelsep=endash,
    parskip=0pt,
    skip=0pt,
    aboveskip=0pt,
    belowskip=0pt
}

%%% Страница

\usepackage{extsizes}        % Возможность сделать 14-й шрифт

\setlength\parindent{1.25cm} % Красная строка 1.25 см

\usepackage{geometry}        % Задать поля
\geometry{top=20mm}
\geometry{bottom=20mm}
\geometry{left=20mm}
\geometry{right=10mm}

\usepackage{fancyhdr}                % Колонтитулы
% \renewcommand{\headrulewidth}{0pt} % Толщина линейки, отчеркивающей верхний колонтитул
% \pagestyle{fancy}
% \lfoot{Нижний левый}
% \rfoot{Нижний правый}
% \rhead{Верхний правый}
% \chead{Верхний в центре}
% \lhead{Верхний левый}
% \cfoot{Нижний в центре}            % По умолчанию здесь номер страницы

\usepackage{setspace} % Интерлиньяж (междустрочный интервал)
\onehalfspacing       % Интерлиньяж 1.5

\usepackage{lastpage} % Узнать, сколько всего страниц в документе.

\usepackage{hyperref} % Гиперссылки

\usepackage[usenames,dvipsnames,svgnames,table,rgb]{xcolor}

\hypersetup{				                         % Настройка гиперссылок
    unicode=true,                                    % русские буквы в раздела PDF
    pdftitle={Практическое задание \textnumero 1},   % Заголовок
    pdfauthor={Александр Рыбин},                     % Автор
    pdfsubject={Практическое задание \textnumero 1}, % Тема
    pdfcreator={Александр Рыбин},                    % Создатель
    pdfproducer={},                                  % Производитель
    pdfkeywords={\LaTeX} {Coursera} {Практика},      % Ключевые слова
    colorlinks=true,       	                         % false: ссылки в рамках; true: цветные ссылки
    linkcolor=red,                                   % внутренние ссылки
    citecolor=green,                                 % на библиографию
    filecolor=magenta,                               % на файлы
    urlcolor=cyan                                    % на URL
}

%%% Работа с графикой

\usepackage{tikz} % Работа с графикой

%%% Разное

\usepackage{titling} % Титульная страница
\usepackage{lipsum} % Вставка "бессмысленного" текста для примеров и отладки

%%% Мета информация



\begin{document} % конец преамбулы, начало документа
    \begin{titlingpage}
        \maketitle
    \end{titlingpage}

    \newpage

    Пусть рис. \ref{fig:fig21} представляет положения Солнца $S$, Земли $T$ и Луны $L$, пусть $\Theta$ есть центр тяжести Земли и Луны. Делаем следующие обозначения:
    \begin{table}[h]
        \centering
        \caption{Обозначения}\label{tab:designations}
        \begin{tabular}{cccc}
           Масса & Солнца & . . . . . & $S$ \\
           >>    & Земли  & . . . . . & $T$ \\
           >>    & Луны   & . . . . . & $L$ \\
        \end{tabular}
    \end{table}

    Расстояние:
    \[S\Theta=\rho;ST=\rho_1; SL=\rho_2; TL=r\]
    тогда будет:
    \begin{equation}
        \begin{aligned}
            T\Theta &= r_1 = \frac{L}{T + L} \cdot r \\
            L\Theta &= r_2 = \frac{L}{T + L} r
        \end{aligned}
    \end{equation}

    Составим теперь выражения ускорений, которые эти тела сообщают друг другу.
    \begin{figure}[h]
        \centering
        \includegraphics{21.png}
        \caption{}\label{fig:fig21}
    \end{figure}

    Солнце $S$ сообщает ускорения:
    \begin{center}
        \begin{tabular}{ccccc}
            Земле & $\displaystyle f \cdot \frac{S}{p_1^2}$ & по & направлению & $TS$ \\[2ex]
            Луне  & $\displaystyle f \cdot \frac{S}{p_2^2}$ & >> & >>          & $LS$ \\
        \end{tabular}
    \end{center}
    вследствие чего точка $\Theta$ имеет ускорения:
    \begin{center}
        \begin{tabular}{ccccc}
            $\displaystyle \frac{L}{T + L} \cdot f \cdot \frac{S}{p_1^2}$ & по & направлению, & параллельному & $TS$ \\[2ex]
            $\displaystyle \frac{L}{T + L} \cdot f \cdot \frac{S}{p_1^2}$ & >> & >> & >> & $LS$ \\
        \end{tabular}
    \end{center}

    Ускорения Солнца, происходящие от Земли и Луны, соответственно суть:
    \begin{center}
        \begin{tabular}{cccc}
            $\displaystyle f \cdot \frac{T}{p_1^2}$ & по  & направлению & $ST$ \\[2ex]
            $\displaystyle f \cdot \frac{L}{p_2^2}$ & >>  & >>          & $SL$ \\
        \end{tabular}
    \end{center}
    поэтому ускорения точки $\Theta$ относительно точки $S$ будут:
    \begin{center}
        \begin{tabular}{ccccc}
            $\displaystyle \omega_1=f \cdot \frac{\left(S + T + L \right)}{T + L} \cdot \frac{T}{p_1^2}$ & по & направлению, & параллельному & $TS$ \\[2ex]
            $\displaystyle \omega_2=f \cdot \frac{\left(S + T + L \right)}{T + L} \cdot \frac{T}{p_2^2}$ & >> & >> & >> & $LS$ \\
        \end{tabular}
    \end{center}

    Разлагая эти ускорения, соответственно, по направлениям $\Theta S$ и $\Theta L$, получим, как легко видеть из подобия показанных на рис. \ref{fig:fig22} и \ref{fig:fig23} треугольников:
    \begin{center}
        \begin{tabular}{cccc}
            $\displaystyle \omega_1^{'} = \omega_1 \cdot \frac{\rho}{\rho_1} $ & по  & направлению & $\Theta S$ \\[2ex]
            $\displaystyle \omega_1^{''} = \omega_1 \cdot \frac{r_1}{\rho_1} $ & по  & направлению & $\Theta L$ \\[2ex]
            $\displaystyle \omega_2^{'} = \omega_2 \cdot \frac{\rho}{\rho_2} $ & по  & направлению & $\Theta S$ \\[2ex]
            $\displaystyle \omega_2^{''} = \omega_2 \cdot \frac{r_2}{\rho_2} $ & по  & направлению & $L \Theta$ \\
        \end{tabular}
    \end{center}
    \begin{figure}[h]
        \centering
        \includegraphics{22.png}
        \caption{}\label{fig:fig22}
    \end{figure}

    Получим для ускорений точки $\Theta$ слагающие:
    \begin{center}
        \begin{tabular}{ccc}
            $\displaystyle W_1 = \omega_1^{'} + \omega_2^{'} = f \cdot \frac{S + L + T}{T + L} \cdot \left[T \cdot \frac{\rho}{\rho_1^3} + L \cdot \frac{\rho}{\rho_2^3} \right] $ & по & $\Theta S$ \\[2ex]
            $\displaystyle W_2 = \omega_1^{''} - \omega_2^{''} = f \cdot \frac{S + L + T}{T + L} \cdot \left[T \cdot \frac{r_1}{\rho_1^3} - L \cdot \frac{r_2}{\rho_2^3} \right] $ & по & $\Theta L$ \\
        \end{tabular}
    \end{center}
    \begin{figure}[h]
        \centering
        \includegraphics{23.png}
        \caption{}\label{fig:fig23}
    \end{figure}

    Заменив $r_1$ и $r_2$ их выражениями \eqref{tab:designations}, имеем:
    \begin{equation*}
        \begin{aligned}
            W_1 = f \cdot \frac{S + T + L}{T + L} \cdot \rho \cdot \left[\frac{T}{\rho_1^3} + \frac{L}{\rho_2^3} \right] \text{ по направлению } \Theta S \\
            W_2 = f \cdot \frac{S + T + L}{\left(T + L\right)^2} \cdot T \cdot L \cdot r \left[\frac{1}{\rho_1^3} - \frac{L}{\rho_2^3} \right] \text{ по направлению } \Theta L
        \end{aligned}
    \end{equation*}
    Но
    \begin{equation*}
        \begin{aligned}
            \rho_1^2 &= \rho^2 + 2\rho \cdot \frac{L}{T + L} \cdot r \cos{\omega} + \left(\frac{L}{T + L} \cdot r \right)^2 \\
            \rho_2^2 &= \rho^2 + 2\rho \cdot \frac{T}{T + L} \cdot r \cos{\omega} + \left(\frac{T}{T + L} \cdot r \right)^2
        \end{aligned}
    \end{equation*}
    следовательно:
    \begin{equation*}
       \begin{aligned}
            \frac{1}{\rho_1^3} &= \frac{1}{\rho^3} \left[1 + 3 \frac{L}{T + L}\cos{\omega} + \left(\frac{L}{T + L} r \right)^2 \left(- \frac{3}{2} + \frac{15}{2} \cos^2{\omega} \right) + \dots \right] \\
            \frac{1}{\rho_2^3} &= \frac{1}{\rho^3} \left[1 + 3 \frac{L}{T + L}\cos{\omega} + \left(\frac{L}{T + L} r \right)^2 \left(- \frac{3}{2} + \frac{15}{2} \cos^2{\omega} \right) + \dots \right]
        \end{aligned}
    \end{equation*}

    Подставляя эти выражения, имеем:
    \begin{equation*}
        \begin{aligned}
            W_1 &= f \cdot \frac{S + T + L}{T + L} \left[1 + \frac{T \cdot L}{\left(T + L \right)^2} \cdot \frac{r^2}{\rho^2} \left(- \frac{3}{2} + \frac{15}{2} \cos^2{\omega} \right) + \dots \right] \\
            W_2 &= f \cdot \frac{S + T + L}{T + L} \left[1 + \frac{T \cdot L}{\left(T + L \right)^2} \cdot \frac{r^2}{\rho^2} \cos{\omega} + \dots \right]
        \end{aligned}
    \end{equation*}

    Но отношения
    \[\frac{L}{T + L} \approx \frac{1}{80}; \frac{r}{\rho} \approx \frac{1}{400}; \left(\frac{r}{\rho} \right)^2 = \frac{1}{160000} \]
    поэтому будет
    \[\frac{T \cdot L}{\left(T + L \right)^2} \cdot \frac{r^2}{\rho^2} \approx \frac{1}{12800000} \]
    и члены, содержащие этот множитель, могут быть отброшены, так что будет:
    \begin{equation*}
        \begin{aligned}
            W_1 &= f \cdot \frac{S + T + L}{\rho^2} \text{ по направлению } \Theta S \\
            W_2 &= 0 \text{ по направлению } \Theta L
        \end{aligned}
    \end{equation*}

    Отсюда следует, что точка $\Theta$ движется вокруг Солнца по элиптической орбите за по законам Кеплера.

    Рассмотрим теперь ускорение Луны по отношению к Земле, для чего к ускорениям, сообщаемым Луне Солнцем и Землёю, надо присовокупить ускорение, равное и противоположное ускорению Земли, происходящему от действия Солнца и Луны. Поступив подобно предыдущему, получим:
    \begin{equation*}
        \begin{aligned}
            f \cdot \frac{T + L}{r^2} + f \cdot S \left[\frac{r_2}{\rho_2^3} + \frac{r_1}{\rho_1^3} \right] \text{ по направлению } L \Theta \\
            f \cdot S \cdot \rho \left[\frac{1}{\rho_2^3} - \frac{1}{\rho_1^3} \right] \text{ по направлению } \Theta S
        \end{aligned}
    \end{equation*}
    положим:
    \[T + L = \mu ; S = M \]

    \listoftables

    \listoffigures
\end{document}